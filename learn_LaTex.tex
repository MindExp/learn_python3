\documentclass[12pt, letterpaper]{article}
\usepackage[]{inputenc}
\usepackage{graphicx}
\usepackage{verbatim}	% we use comment command to excute mutiple comments in latex.
\graphicspath{C:/Users/Administrator/Desktop}	% This line seems not working.

\title{First document}
\author{Hubert Farnsworth}
\date{July 2018}

\begin{document}
\maketitle

\tableofcontents
First document. This is a simple example, with no 
extra parameters or packages included.

% This \textbf{•} here is a comment. It will not be printed in the document.
\textbf{Bold, italics and underlining}

Some of the \textbf{greatest} discoveries in \underline{science} were made by \textbf{\textit{accident}}.

% What the \emph command actually does with its argument depends on the context.
Some of the greatest \emph{discoveries} in science were made by accident.
\textit{Some of the greatest \emph{discoveries} in science were made by accident.}
\textbf{Some of the greatest \emph{discoveries} in science were made by accident.}

\textbf{Adding images}

\centering
\includegraphics[width=3.5in]{./src/CNN_book_tutorial_wxs/2018-06-26_222331}	% Chinese can not be contained in the file path.

\begin{figure}[ht]
	\centering
	\includegraphics[scale=0.6]{./src/CNN_book_tutorial_wxs/2018-06-27_164120}
	\caption{a nice plot}	%  you may expect this command sets the caption for the Ògure.
	\label{fig:plot}	%  If you need to refer the image within your document, set a label with this command.
\end{figure}

As you can see in the figure \ref{fig:plot}, It is an example of sigmoid function. Also, in the page \pageref{fig:plot} is the same example.

\textbf{Unordered lists}

\begin{itemize}
	\item The individual entries are indicated with a black dot, a so-called bullet.
	\item The text in the entries may be of any length.
\end{itemize}

\textbf{Ordered lists}

\begin{enumerate}
	\item This is the first entry in our list
	\item The list numbers increase with each entry we add
\end{enumerate}

\textbf{Adding math to LaTeX}

\begin{itemize}
	% To put your equations in inline mode use one of these delimiters: \( ... \), $ ... $ or \begin{math} ... \end{math}. They all work and the choice is a matter of taste.
	\item \textbf{Inline model:}
	In physics, the mass-energy equivalence is stated by the equation $E=mc^2$, discovered in 1905 by Albert Einstein.
	
	% The displayed mode has two versions: numbered and unnumbered.
	% To print your equations in display mode use one of these delimiters: \[ ... \], $$ ... $$,\begin{displaymath} ...\end{displaymath} or \begin{equation} ... \end{equation}
	\item \textbf{Display model:} 
	The mass-energy equivalence is described by the famous equation
	$$E=mc^2$$
	discovered in 1905 by Albert Einstein. In natural units ($c = 1$), the formula expresses the identity
	% Important Note: equation* environment is provided by an external package.
	\begin{equation}
		E=m
	\end{equation}
\end{itemize}

Subscripts in math mode are written as $a_b$ and superscripts are written as $a^b$.
These can be combined an nested to write expressions such as
$$T^{i_1 i_2 \dots i_p}_{j_1 j_2 \dots j_q} =
T(x^{i_1},\dots,x^{i_p},e_{j_1},\dots,e_{j_q})$$

We write integrals using $\int$ and fractions using $\frac{a}{b}$. Limits are placed on integrals using superscripts and subscripts:
$$\int_0^1 \frac{1}{e^x} = \frac{e-1}{e}$$

Lower case Greek letters are written as $\omega$ $\delta$ etc. while upper case Greek letters are written as $\Omega$ $\Delta$.

Mathematical operators are prefixed with a backslash as $\sin(\beta)$, $\cos(\alpha)$, $\log(x)$ etc.

\textbf{Basic Formatting}

\textbf{\textit{Abstracts}}
\begin{abstract}
This is a simple paragraph at the beginning of the document. A brief introduction about the main subject.
\end{abstract}

Now that we have written our abstract, we can begin writing our first paragraph.

% When writing the contents of your document, if you need to start a new paragraph you must hit the "Enter" key twice (to insert a double blank line). Notice that LaTeX automatically indents paragraphs.
This line will start a second Paragraph.
%To start a new line without actually starting a new paragraph insert a break line point, this can be done by \\ or \newline command.
\\ This will start a new line.
\newline This will start a another line.

% Commands to organize a document vary depending on the document type, the simplest form of organization is the sectioning, available in all formats.

% \chapter{Chapter}		% chapter syntax not working in article environment,  \part and \chapter are only available in report and book document classes.

\section{First Section}
This is the first section.

Lorem ipsum dolor sit amet, consectetuer adipiscing elit. Etiam lobortisfacilisis sem. Nullam nec mi et neque pharetra sollicitudin. Praesent imperdietmi nec ante. Donec ullamcorper, felis non sodales...

\section{Second Section}
Lorem ipsum dolor sit amet, consectetuer adipiscing elit. Etiam lobortis facilisissem. Nullam nec mi et neque pharetra sollicitudin. Praesent imperdiet mi necante... 

\subsection{First Subsection}
Praesent imperdietmi nec ante. Donec ullamcorper, felis non sodales...

\subsection*{Unnumbered Section}
Lorem ipsum dolor sit amet, consectetuer adipiscing elit. Etiam lobortis facilisissem.

% we could multiple comments in latex as follow
\begin{comment}
The basic levels of depth are listed below:
-1 \part{part}
0 \chapter{chapter}
1 \section{section}
2 \subsection{subsection}
3 \subsubsection{subsubsection}
4 \paragraph{paragraph}
5 \subparagraph{subparagraph}
\end{comment}

\textbf{Creating a simple table in LaTeX}

\begin{table}
\begin{center}
\begin{tabular}{| c | c | c |}		% specify the parameter of tabular environment.
	% we can add borders using the horizontal line command \hline and the vertical line parameter |.
	\hline	% horizontal line
	cell1 & cell2 & cell3 \\
	\hline
	cell4 & cell5 & cell6 \\
	\hline\hline
	cell7 & cell8 & cell9 \\
	\hline
\end{tabular}
\end{center}
\end{table}

Table \ref{table:data} is an example of referenced \LaTeX{} elements, and it is in page \pageref{table:data}.

\begin{table}
\begin{center}
\begin{tabular}{||c c | c c||}
	\hline
	Col1 & Col2 & Col2 & Col3 \\ [0.5ex]
	\hline\hline
	1 & 6 & 87837 & 787 \\
	\hline
	2 & 7 & 78 & 5415 \\
	\hline
	3 & 545 & 778 & 7507 \\
	\hline
	4 & 545 & 18744 & 7560 \\
	\hline
	5 & 88 & 788 & 6344 \\ [1ex]
	\hline
\end{tabular}
\caption{Table to test captions and labels}	% description of the table.
\label{table:data}	% the unique lable of the table.
\end{center}
\end{table}

\thanks{funded by the ShareLaTeX team}
\today
\end{document}
